\section{Complétude}
\subsection{Critère de Cauchy}

\begin{definition}
Une suite $(x_{n})$ dans un espace métrique $(X,d)$ est de Cauchy si et seulement si :
$$\forall \varepsilon >0,\ \exists N\in \mathbb{N} ,\ \forall p,q\ge N,\ d(x_p,x_q)\le \varepsilon $$\\
De manière équivalente : $d(x_p,x_q)\le \varepsilon _{\min p,q}$ avec $\varepsilon _n \xrightarrow[n\to +\infty]{} 0$.\\

\end{definition}

Une suite de Cauchy :
\begin{itemize}
    \item est toujours bornée : $d(x_0,x_n)\le \varepsilon _0$
    \item admet au plus une valeur d'adhérence
    \item si elle admet une valeur d'adhérence alors elle converge vers celle ci
\end{itemize}
Toutes suites convergente est de Cauchy.

\begin{definition}
    $(X,d)$ est complet si et seulement si toutes suites de Cauchy converge
\end{definition}
\begin{lemme}
    Soit $(X,d)$ complet, $A \subset X$ alors $(A,d_{|A\times A})$ est complet si et seulement si $A$ est fermé
\end{lemme}


\begin{remarque}
\begin{itemize}
    \item un evn complet est appelé un (espace de) Banach. \\
    
    \item un evtlc complet pour la distance associée est appelé un (espace de) Fréchet.
\end{itemize}
\end{remarque}

\begin{lemme}[Série dans un Banach] 

Soit $(E,\|.\|)$ un evn. Sont équivalents :
    \begin{itemize}
        \item $E$ est complet
        \item toute série absolument convergente (ie $\sum\limits_{n=1}^{\infty} \|y_n\|<\infty $) est convergente.
    \end{itemize}
\end{lemme}
\begin{proof}
    Supposons $E$ complet. Soit $(y_n)$ le terme général d'une série absolument convergente. 
    On définit $x_N:=\sum\limits_{n\le N}^{} y_n$, $\varepsilon _n:=\sum\limits_{n>N}^{} \|y_n\|$. 
    Alors $\varepsilon _n \xrightarrow[n\to +\infty]{} 0$ comme reste d'une série sommable, et $\forall p\le q,\ \\
    \|x_p-x_q\|=\|\sum\limits_{r=p+1}^{q} y_r\|\le \sum\limits_{r=p+1}^{q} \|y_r\|\le \varepsilon _p = \varepsilon_{\min p,q} $ donc les sommes partielles satisfont le critère de Cauchy donc convergent.\\
    
    \textbf{Réciproquement :} si $(x_{n})$ de Cauchy, $\|x_p-x_q\|\le \varepsilon _{\min p,q}$ où \\
    $\varepsilon _N\to 0$. Soit $(N_k)_k$ strictement croissante telle que $\varepsilon _{N_k}\le 2^{-k}$. Posons $y_k:=x_{N_{k+1}}-x_{N_k}$.
    La série des $y_k$ est sommable donc converge par hypothèse donc $\sum\limits_{k<K}^{} y_k=x_{N_k}-x_{N_0}$ converge. Donc $x_{n}$ est une suite de Cauchy admettant une valeur d'adhérence donc elle est également convergente.
\end{proof}


\subsection{Exemple d'espaces fonctionnels complets}
\begin{ex}[Fonctions bornées] : soit $(X,d)$ espace métrique $E$ de Banach. Alors $C^0_b(X,E)$ est complet pour norme $\infty $.
\end{ex}
\begin{proof}
    Soit $(f_n)$ de Cauchy. $|f_p(x)-f_q(x)|\le \|f_p-f_q\|_\infty \le \varepsilon _{\min p,q}$. Donc $(f_n(x))$ de Cauchy et admet une limite $f_\infty (x)$. De plus $\|f_p-f_\infty \|\le \varepsilon _p$. Enfin $f_\infty $ est continue (resp bornée) comme limite uniforme d'une suite de fonction continues.
\end{proof}

\begin{ex}[Espaces $L^p$] : soit $p \in [1,\infty ]$, $(X,d)$ espace mesuré, alors $L^p$ est complet.
\end{ex}
\begin{proof}
    $\exists $ classes d'équivalences modulo égalité pp. Soit $(f_n)$ une série sommable. Posons $S_N(x):=\sum\limits_{n\le N}^{} |f_n(x)|$ et $S_\infty $ la limite (possiblement $\infty $). Alors $\left( \int_XS_N(x)^pdx \right) ^{\frac{1}{p}}=\|S_N\|_p\le \sum\limits_{nn\le N}^{} \|f_n\|_p\le C<\infty$. D'où $\int_XS_\infty (x)^pd\mu(x)=\lim\limits_{} \int_XS_N(x)^pd\mu(x)\le C^p<\infty $. Par le th de convergence monotone (Boffo Levi) car $S_N(x)\searrow S_\infty(x)$. Donc $S_\infty <\infty $ pour $\mu- pp\ x$. On pose alors $g_\infty (x):=\sum\limits_{n\in \mathbb{N} }^{} f_n(x)$ qui est convergente $\mu pp\ x$. On pose aussi $g_N(x)$ la somme partielle. Alors $|g_\infty(x) -g_N(x)|\le \sum\limits_{n>N}^{} |f_n(x)|$ donc $\|g_\infty -g_N\|_p\to 0$.
\end{proof}

\begin{ex}[Fonctions bornées] : soit $k\in \mathbb{N} $, $\Omega\subset \mathbb{R} ^d$ ouvert, alors $C^k_b(\Omega)$ est un Banach pour la norme $\|f\|:=\sum\limits_{|\alpha |\le k}^{} \|\partial_{\alpha }f\|_\infty $.
\end{ex}
\begin{proof}
    Soit $(f_n)$ de Cauchy et $(f_n^\alpha)=\partial_\alpha f_n$. Alors c'est aussi de Cauchy dans $C^0_b(X)$ donc cv vers $f^\alpha $. Soit $\alpha \in \mathbb{N} ^d$ avec $|\alpha |< k$ $x\in \Omega,\ 1\le i\le d$. Justifions que $\partial/\partial_{x_i}f^\alpha (x)=f^{\alpha+e_i}(x) $ avec $e_i$ la base canonique. Soit $p>0$ tq $[x,x+pe_i]\subset \Omega$, alors $f^\alpha _n(x+pe_i)-f^\alpha _n(x)=\int_0^pf_n^{\alpha +e_i}(x+te_i)dt$ car $\frac{\partial}{\partial_{x_i}}f_n^\alpha =f_n^{\alpha +e_i}$.\\
    Par cv uniforme, on a pareil mais sans \\
    $f^\alpha(x+pe_i)-f^\alpha(x)=\int_0^pf^{\alpha}(x+te_i)dt$ continument dérivable / p.\\
    Finalement $\|f_n-f^0\|=\sum\limits_{|\alpha |\le k}^{} \|\partial_\alpha f_n-\partial_\alpha f^0\|=\sum\limits_{}^{} \|f_n^\alpha -f^\alpha \| \xrightarrow[n\to +\infty]{} 0$ car $\partial _\alpha f^0=f^\alpha $.
\end{proof}

\begin{ex}
    Soit $\Omega$ ouvert et $(\Omega_n\neq \emptyset )$ tq $\bigcup\limits_{n\in \mathbb{N} } \Omega_n=\Omega$ et $\overline{\Omega}\subset _C\Omega_{n+1}$. Soit $k\in \mathbb{N} \cup \{\infty \} $, alors $\left( C^k(\Omega),(|.|_{n,\alpha })_{n\in \mathbb{N} }^{|\alpha |\le k} \right) $ est un Fréchet.
\end{ex}
\begin{proof}
    (cas $k=\infty $). Soit $(f_n)$ de Cauchy. Soit $k'\in \mathbb{N} $ arbitraire (on prendrait $k'\le k$ dans le cas $k<\infty $). Alors $(f_{n|\Omega_n}$ est une suite de Cauchy de $C_b$. Or elle admet une limite $g'^k_n$sur $\Omega_n$.
\end{proof}


\begin{ex}
    $C^\infty _b(\Omega)$ muni de $(\|.\|_n)_n$   où $\|f\|_n:=\max_{|\alpha |\le n}\|\partial _\alpha f\|_{\infty }$ est Fréchet.
\end{ex}
\begin{proposition}
    $\mathcal{D}_k(\Omega)$   où $k\subset _C\Omega$, compact et $\Omega$  ouvert. $\mathcal{D}_K(\Omega):=\{f\in \mathcal{D}(\Omega)|supp(f)\subset K\} $  est un espace fermé de l'ensemble initial. De plus la topologie induite sur $\mathcal{D}_K(\Omega)$ par $(\mathcal{D}(\Omega),(|.|_{w, \eta}$) et $(C^\infty _b(\Omega),\cdots)$ est la même.
\end{proposition}
\begin{proof}
    Fermeture : Si $f_n \xrightarrow[n\to +\infty]{} f$ avec $f_n\in \mathcal{D}_\alpha (\Omega)$ pour la topo $C^\infty _c$ alors en particulier $f_n \xrightarrow[n\to +\infty]{} f$ uniformément donc $supp(f)\subset K$.\\
    Posons $supp(f):=\overline{\{x\in \Omega|f(x)\neq 0\}}$.\\
    Mêmes topologies suivantes : $\|f\|_n\le |f_{w, \eta}|$ en prenant $w=1, \eta=n$. $|f|_{w, \eta}|\le C\|f\|_n, \forall f\in \mathcal{D}_K(\Omega),$ en prenant $C=\max_{x\in K}w(x)$, on peut borner les semi normes d'une famille par une cte $x$ un max d'un nombre fini de semi normes de l'autre donc les mêmes topos.
\end{proof}
\begin{proposition}
    Soit $\varphi $  une forme linéaire sur $\mathcal{D}(\Omega)$. Sont équivalent :
    \begin{itemize}
        \item $\varphi $  est continue sur $\mathcal{D}(\Omega)$, ie $\exists w, \eta\in C^\infty (\Omega,\mathbb{R} ^+),\ \forall f\in \mathcal{D}(\Omega),\ |\varphi (f)|\le |f|_{w, \eta} $
        \item $\varphi $  est continue sur $D_K(\Omega)$  ie $\forall K\subset _C\Omega,\ \exists w_k, \eta_K\in \mathbb{R} ^+\times \mathbb{N} ,\ \forall f\in \mathcal{D}_K(\Omega),\ |\varphi (f)|\le w_K\|f\|_{\eta_K}  $
    \end{itemize}
    De plus, $\varphi $  est d'ordre fini $k\in \mathbb{N} $  ssi on peut choisir $\eta=k$, de manière équivalente, $\eta_K=k,\forall K\subset _C\Omega $.
\end{proposition}
\begin{remarque}
    On dit que $\mathcal{D}(\Omega)$ est la limite inductive des $\mathcal{D}_K(\omega)$
\end{remarque}

\begin{lemme}[Quelques fonctions $C^\infty $]

Les fonctions suivantes sont $\mathcal{C}^\infty$ :
\begin{enumerate}
        \item La fonction $\psi_0:
            \begin{aligned}
                \mathbb{R} &\to \mathbb{R} \\
                x&\mapsto 0 \text{ si } x<0, e^{-\frac{1}{x}} \text{ sinon}
            \end{aligned}$
        \item La fonction $\psi_1x\mapsto \int_0^x\psi_0(t)\psi_0(1-t)dt$ est $C^\infty $, vaut 0 sur $]-\infty ,0]$  vaut une constante sur $[1,\infty [$. $H:=\frac{\psi_1}{\psi_1(1)} $  est une application de la fonction de Heaviside
        \item La fonction $\psi_2:x\in \mathbb{R} ^d\mapsto \psi_0(1-\|x\|^2)$ est $C^\infty $  positive, radiale, à support égal à $B'_{\mathbb{R} ^d}(0,1)$. Souvent utilisée comme noyau de convolution pour régulariser les filtres.
        \item Soit $K\subset_CU$, $K$ compact, $U\subset \mathbb{R} ^d$ ouvert. Alors $\exists \psi\in C^\infty (\mathbb{R} ^d),$, $\psi=1$ sur $K$et $supp(f)\subset U$
    \end{enumerate}
\end{lemme}
\begin{proof}
    \begin{enumerate}
        \item Classique
        \item facile
        \item facile
        \item $\forall x\in K, $  soit $r_x>0$ tq $B(x,r_x)\subset U$. On extrait un sous recouvrement fini de $K\subset \bigcup\limits_{x\in K} B(x,\frac{r_x}{3})$, noté $K\subset \bigcup\limits_{1\le i\le I} B(x_i, \frac{r_i}{3})$. Posons $\varphi (x):=\sum\limits_{1\le i\le I}^{} \psi_2(\frac{x-x_i}{r_i /2})$. Alors $\psi_2(\frac{x-x_i}{r_i /2})>0$ sur $B(x_i,\frac{r_i}{3})$ et son support (supp) sur $B'(\cdots)$. Donc $\varphi >0$  sur $\bigcup\limits_{1\le i\le I} B(x_i,\frac{r_i}{3})\supset K$. $supp(\varphi )=\bigcup\limits_{1\le i\le I} B'(x_i,\frac{r_i}{3})\subset _CU$.\\
            Par compacité, $\varepsilon :=\min_{x\in K}\varphi (x)$  est strictement positif. On considère finalement $\psi:=H\circ \varphi $. Où $H\in C^\infty (\mathbb{R} ,\mathbb{R} ), H=0$ sur $]-\infty ,0],$ $H=1$ sur $[\varepsilon ,\infty [$ satisfait $supp(\psi)\subset supp(\varphi )\subset _CU$ et \\
            $\psi^{-1}([\varepsilon ,\infty [)\supset \varphi ^{-1} ([\varepsilon ,\infty [$.
    \end{enumerate}
\end{proof}

\begin{lemme}
Soit $f,g\in C^\infty (\mathbb{R} ^d), \alpha \in \mathbb{N} ^d$  alors $\partial_\alpha (fg)=\sum\limits_{\beta\le \alpha }^{} {\alpha \choose \beta} \partial_\beta f\partial_{\alpha -\beta}g$ où ${\alpha \choose \beta}:=\prod\limits_{1\le i\le d}^{} {\alpha _i \choose \beta_i}$
\end{lemme}
\begin{proof}
    Cas où $\alpha =(n,0,\cdots,0)$ alors $\frac{\partial^n}{\partial x_1^^n}(fg)=\sum\limits_{0\le k\le n}^{} {n \choose k} \frac{\partial^k}{\partial x_i^k}f\frac{\partial^{n-k}}{\partial x_i^{n-k}}g$ par récurrence immédiate.\\
    Passage de $(\alpha _1,\cdots, \alpha _{k-1},0\cdots,0)=\alpha _*$ à $(\alpha _1,\cdots\,a_k,0\cdots,0)$. Récurrence sur $k$.\\
    $\partial_{\alpha }(fg)=\frac{\partial^{\alpha _k}}{\partial x_k^{\alpha _k}}=\sum\limits_{\beta_*\le \alpha _*}^{} {\alpha _* \choose \beta_*}\cdots$ Par HR et linéarité de la dérivation. \\
    Puis on utilise ${\alpha _0 \choose \beta_0}{\alpha _k \choose \beta_k}={\alpha \choose\beta}$ et le résultat tombe.
\end{proof}
\begin{proof}
    (Critère de continuité des distributions) : Soit $(\Omega_n)$  tq $\overline{\Omega_n}\subset _C\Omega_{n+1}$, tous ouvert et formant une partition de $\Omega$. Soit $(\gamma_n)$  tq $\gamma_n\in C^\infty , \gamma_n=1$ sur [?,?] et $supp(\gamma_n)\subset \Omega_n$. Supp (ii : $\mathcal{D}_K(\Omega)\cdots$  ). Soit $w_n, \eta_n$
    tq $\forall f\in \mathcal{D}(\Omega),\ supp(f)\subset \overline{\Omega}\Rightarrow |\varphi (f)|\le w_n\|f\|_{\eta_n}. $  Soit $f\in \mathcal{D}(\Omega).$  Alors $f=\sum\limits_{n\in \mathbb{N} }^{=} f(\gamma_n-\gamma_{n+1}:=\beta_n)$ avec $\gamma_{-1}=0$. De plus cette somme a un nombre fini de termes non nuls.
    En effet, $\exists N,\ \forall n\ge N,\ supp(f)\subset \Omega_n, $ par compacité de supp(f). Donc $\forall n\ge N+1,f(\underbrace{\gamma_n-\gamma_{n-1}}_{\text{nul sur }\overline{\Omega_{n-1}}})=0$
    Par linéarité, $|\varphi (f)|\le \sum\limits_{n\in \mathbb{N} }^{} |\varphi (f_{\beta_n})|\le \sum\limits_{n\in \mathbb{N} }^{} w_n\|f\beta_n\|_{\eta_n}$
    (car $supp(\beta_n)\subset \Omega_{n+1}\backslash \Omega_{n-1})\le \sum\limits_{n\in \mathbb{N} }^{} w_{n+1}$
$\sup_{\alpha \le  \eta_{n+1}, x\in \Omega_{n+1}\backslash \Omega_{n-1}}|\partial_\alpha (f\beta_n)(x)|\le \sum\limits_{n\in \mathbb{N} }^{}$
$\underbrace{\tilde{w_{n+1}}}_{\text{dépend des }\|\partial_{\alpha , \beta_n}\|}\sup\cdots(paseuletempsd'ecrire)\le \sup_{n\in \mathbb{N}, \alpha \le \eta_{n+1},x\in \Omega_{+1}\backslash \Omega_{n-1} }\hat{w_{n+1}}|\partial_\alpha f(x)|$
    avec $\hat{w_{+1}}:=C_\alpha (1+n)^\alpha \tilde{w_{n+1}}$ \\
$\le |f|_{w, \eta}$  où $w, \eta$  vérifient $w(x)\ge w_n$ si $x\not\in \Omega_{n-2}, \eta(x)\ge \eta_n$ si de même.\\
Par ex $w(x)=\sum\limits_{n\in \mathbb{N} }^{\infty} w_n(\underbrace{1-\gamma_{n-3}}_{\text{vaut 1 hors de $\Omega_{n-2}$}})$
\end{proof}



\begin{ex}
    Soit $(X,d)$ un espace métrique, $w$ un module de continuité strictement positif hors de 0. Posons $\forall f\in C^\infty _b(X),\\ |f|_w=\sup_{x,y\in X}\frac{|f(x)-f(y)|}{w(d(x,y))},\|f\|_w:=|f|_w+\|f\|_\infty $. \\
    Alors $\{f\in C^0_b(X)|\|f\|_w<\infty \} $ est un Banach.\\
    Cas particulier : fct Lipschitziennes bornées / Hölderienne bornées.
\end{ex}

\subsection{Prolongements :}
\begin{propriete}[Prolongement des fonctions uniformément continues] : Soit $X,Y$ des espaces métriques complets, $A\subset X$ une partie dense, $f:A\to Y$ uniformément continue. Alors $f$ admet une unique extension continue $F:X\to Y$ (qui se trouve être uniformément continue).
\end{propriete}
\begin{proof}
    \textbf{Construction} : on def $f(x):=\lim f(x_n)$ où $x_{n}\in A$ et $x_{n}\to x$. $(x_{n})cv\Rightarrow (x_{n})$ est de Cauchy $\Rightarrow f(xn)$ est de Cauchy $\Rightarrow f(x_{n})cv$ \\
    \textbf{Bonne définition} : Si $x_{n},y_n \to x,x$ alors $d(x_{n},y_n)\to 0$ donc \\$d(f(x_{n}),f(y_n))\to 0$ par uniforme continuité de $f$. Finalement \\$\lim f(x_{n})=\lim f(y_n).$ \\
\textbf{Continuité uniforme} : supposons $x_{n}\to x, y_n\to y$ alors $d(\lim f(x_{n}), \lim f(y_n))=\lim d(f(x_{n}),f(y_n))\le \lim w(d(x_{n},y_n))=w(d(x,y))$. On peut supposer $w$ continue donc le résultat tombe.\\
    Unicité : parmi les fct continues, découle de la construction.
\end{proof}
\begin{remarque} [Extension de Tietze] : Si $f$ uniformément continue sur $A\subset X$ qcq, on a toujours une extension à priori pas unique. OPS(on peut supposer) $w$ croissant et sous additif. $F(x):=\inf_{y\in A}f(y)w(d(x,y))$
\end{remarque}
\begin{remarque}
    En pratique $X$ et $Y$ sont souvent des Banach, $f$ est une application linéaire continue de $A\subset X$ dense dans $Y$.
\end{remarque}
\begin{propriete}[Complété d'un espace] 

Soit $(A,d)$ un espace métrique. Alors il existe $(X,d)$ métrique, complet et une injection isométrique $i_A:A\to X$ tq $Im(i_A)$ est dense dans $X$. De plus $X$ est unique à isométrie près.
\end{propriete}
\begin{proof}
    Existence : $X=\{\text{suites de Cauchy de }A\} /\sim$ où $(x_{n})\sim(y_n)\Leftrightarrow d(x_n,y_n)\to 0$. \\
    Unicité : découle du résultat d'extension précédent :
    % https://q.uiver.app/#q=WzAsMyxbMCwxLCJBIl0sWzIsMCwiWCJdLFsyLDIsIlgiXSxbMCwxLCJpX0EiXSxbMCwyLCJcXHRpbGRle2lfQX0iLDJdXQ==
$\begin{tikzcd}
    A && X\\
        && \tilde{X}
        \arrow["{i_A}", from=1-1, to=1-3]
        \arrow["{\tilde{i_A}}"', from=1-1, to=2-3]
    \arrow["\varphi", dotted, from=1-3, to=2-3]
\end{tikzcd}$
Alors $\varphi :\begin{aligned}Im(i_A)&\longrightarrow Im(\tilde{i_A}) \\ x&\longmapsto \tilde{i_A}(i_A^{-1} (x))\end{aligned}$ est une isométrie sur une partie dense de $X$ donc s'étend uniquement en une isométrie de $X\to \tilde{X}$
\end{proof}

\subsection{Point fixes de Picard}
\begin{propriete}
    Soit $(X,d)$ métrique complet, $f:X\to X$, $K$-lipschitzienne avec $K<1$ (ie contractante). Alors $f$ a un unique point fixe $x_*$. De plus $\forall x_0\in X,\ d(x_0,x_*)\le \frac{d(x_0,f(x_0))}{1-K}.$
\end{propriete}
\begin{proof}
    Unicité : Si $x_*$ et $\tilde{x_*}$ sont des points fixes, $d(x_*,\tilde{x_*})=d(f(x_*),f(\tilde{x_*}))\le K\cdots<d(x_*,\tilde{x_*})$ donc $d(x_*,\tilde{x_*})=0$.\\
    Extension et estimation : soit $x_0\in K$ puis $x_{n+1}=f(x_{n})$ alors $d(x_{n}, x_{n+1})\le Kd(x_{n-1},x_{n})\le K^nd(x_0,x_1)$. Ainsi pour $p\le q\cdots$
    Donc $(x_{n})$ satisfait le critère de Cauchy donc cv vers une limite $x_*.$ $d(x_N,x_*)\le K^N\frac{d(x_0,x_1)}{1-K}$. Ainsi $d(x_*,f(x_*))=\lim d(x_{n},x_{n+1)})=0$.
\end{proof}
\begin{remarque}
    (Stabilité) : Si $f$ est $K-$lipschitzienne avec $K<1$, si $\|f-g\|_\infty \le \varepsilon $ et si $x_\varepsilon $ est un point fixe de $g$, alors $d(\underbrace{x_\varepsilon }_{\text{pt fixe de }g},\underbrace{x_*}_{\text{pt fixe de }f})\le \frac{d(x_\varepsilon ,f(x_\varepsilon ))}{1-K}\le \frac{\varepsilon}{1-K}$
\end{remarque}
\begin{theoreme}[Cauchy Lipschitz] 

Soit $\Omega\subset \mathbb{R} ^d$ ouvert. Soit $f:\mathbb{R} ^+\times \Omega\to \mathbb{R} ^d$ continue et localement lipschitzienne en sa seconde variable ie $\forall T\ge 0,\ \forall K\subset _C\Omega,\ \exists C=C(T,K),\ \forall t\in [0,T],\ \forall x,y\in K,\  \|f(t,x)-f(t,y)\|\le C\|x-y\|$. Alors $\forall x\in \mathbb{R} , $ il existe $t_*>0$ et $u:[0,t_*]\to \Omega$ tq $u(0)=x_0$ et $u'(t)=f(t,u(t))$.
\end{theoreme}
\begin{remarque}
    Une propriété $P:\mathcal{P}(\Omega)\to \{\text{Vrai, Faux}\} $ est satisfaite localement ssi tout point $x\in \Omega$ admet un voisinage $V\in \mathcal{V}_x$ tq $P(V)$ est vrai. \\Si $\Omega$ est localement compact (vrai si $\Omega\subset \mathbb{R} ^d$), (tt pt admet une base de voisinage compact) et $(P(A)\wedge P(B))\Rightarrow P(A\cup B)$, $(P(A)\wedge  B\subset A))\Rightarrow P(B)$ alors $P$ est satisfaite localement ssi elle est satisfaite sur tout compact.
\end{remarque}
\begin{proof}
    Preuve de l'existence dans CL : Soit $r_0>0$ tq $B'(x_0,r_0)\subset \Omega$. Soit $t_0>0$ alors $f$ est bornée par $C^\infty $ sur $[0,t_*]\times B'(x_0,r_0)$ et $f$ est $C_{lip}$ lipschitzienne sur le même intervalle.\\
    Définissions $t_1>0$ tq $C_\infty t_1<r_0$ et $C_{lip}t_1<1$. Posons $X=C^0([0,t_1],B'(x_0,r_0))$ complet. $F:X\to X$ tq $F(u)=F_u:[0,t_1]\to B'(x_0,r_0)$ avec $F_u(t)=x_0+\int_0^{t_1}C_\infty ds\le t_1C_\infty \le r_0.$ \\
    Caractère contractant : $\forall u,v\in X,\ \|F_u(t)-F_v(t)\|\le \int_0^{t_1} \|f(s,u(s))-f(s,v(s))\|ds\le \int_0^{t_1}C_{lip}\|u(s)-v(s)\|ds\le C_{lip}t_1\|u-v\|_\infty .$ Donc les conditions du point fixe de Picard sont réunies. $F$ admet un point fixe qui est par contraction $C^1$ et par dérivation est solution du pb de Cauchy :)
\end{proof}
\begin{remarque}
    Le pt fixe de Picard implique aussi la stabilité par rapport aux conditions initiales. Cependant on le montre en général en utilisant le lemme de Gronwall, un peu plus précis
\end{remarque}

\begin{lemme}[Gronwall]

    Soit $f\in C^0([0,T],\mathbb{R} ^+)$ et $A,B\ge 0$ tq $\forall t\in [0,T],\ f(t)\le A\underbrace{\int_0^tf(s)ds}_{=:F(t)}+B $. Alors $f(t)\le Be^{-At}$.
\end{lemme}
\begin{proof}
    On a $F'(t)=Af(t)\le AF(t)$ donc $\left( F(t)e^{-At} \right)'=\left( F'-AF \right) e^{-At}\le 0$. Donc $F(t)e^{-At}$ est décroissante en $t$. Donc $F(t)e^{-At}\le F(0)=B$. Donc $f(t)\le F(t)\le Be^{-At}$
\end{proof}
\begin{propriete}[Stabilité dans CL]

Sous les hypothèses $f:R\times \Omega\to \mathbb{R} ^d$ continue, localement lipschitzienne selon la seconde variable. Soit $u,v\in C^1([0,T],K)$ solution de $u'(t)=f(t,u(t))$ où $K\subset _C\Omega.$ Alors $\|u(t)-v(t)\|\le e^{Ct}\|u(0)-v(0)\|$ avec $C=C(T,K)$ constante de Lipschitz.\\
\end{propriete}
\begin{proof}
    $\|u(t)-v(t)\|=\|\int_0^t(u'(s)-v'(s))ds+(u(0)-v(0)\|$ car $u(t)=u(0)+\int_0^tu'(s)ds$. Donc $\le \|\int^t_0(f(s,u(s))-f(s,v(s)))ds\|+\underbrace{\|u(0)-v(0)\|}_{=:B}$ $\le \overbrace{C}^A\int_0^t\|u(s)-v(s)\|ds+B$ le résultat s'obtient par Gronwall appliqué à $u-v$.
\end{proof}
\begin{ex}[EDO avec retard]

Il existe une unique solution $\nu\in C^1([0,1],\mathbb{R} )$ de $\left\{
        \begin{tabular}{c c}
            $\nu(0)=1$\\
            $\nu'(t)=\nu(t-t^2)$
\end{tabular}\right.$
\end{ex}
\begin{proof}
On cherche un point fixe de $F:X\to X$ définit comme avant.  $|F_u(t)|\le 1+\int_0^{\frac{1}{2}}4=3$ donc $F$ bien def et $F_u$ positive.\\
$|F_u(t)-F_v(t)|\le \int_0^{\frac{1}{2}}|u(t-t^2)-v(t-t^2)|dt\le \frac{1}{2}\|u-v\|_\infty $
\end{proof}
\begin{ex}
    Soit $k\in C^0([0,1]^2,]-1,1[)$ et  $\varphi \in C^0([0,1],\mathbb{R} )$ alors il existe une unique sol de $u(t)=\underbrace{\int_0^1\underbrace{k(s,t)}_{\le K<1}\underbrace{\frac{u(s)}{1+u^2(s)}}_{r\mapsto \frac{r}{1+r^2}\text{ est lipschitzienne}}ds}_{F_u(t)}$. D'où $|F_u(t)-F_v(t)|\le K\|u-v\|_\infty $ et $F$ est contractante sur cette topologique.
\end{ex}

\subsection{Théorème de Baire}
\begin{lemme}[Fermés emboités] : Soit $(X,d)$ un espace métrique complet et $(F_n)$ une suite de fermés de $X$ tq $F_{n+1}\subset F_n$ et $diam(F_n)\to 0.$ $diam(F_n):=\sup_{x,y\in F_n}d(x,y)$. Alors $\bigcap\limits_{n\in \mathbb{N} } F_n=\{x_*\} $ pour un certain $x_*\in X.$
\end{lemme}
\begin{proof}
    Soit $x_{n}\in F_n$ arbitraire. Alors $\forall N,\ \forall p,q\ge N,\ d(x_p,x_q)\le diam(F_N).  $ donc $(x_{n})$ est de Cauchy. Sa limite $x_*$ appartient à chaque disque $F_n$ par fermeture donc $x_*\in \cap F_n.$ De plus si $y_*\in \cap F_n$ alors $\forall n,\ d(x_{n},y_*)\le diam(F_n)\to 0 $ donc $x_*=y_*$.
\end{proof}
\begin{theoreme}[Baire]

    Soit $(X,d)$ mesuré et $(U_n)$ une suite d'ouverts denses. Alors $\bigcap\limits_{n\in \mathbb{N} } U_n$ est dense.
\end{theoreme}
\begin{proof}
    Soit $x_0\in X, \varepsilon _0 >0$ arbitraire. $B(x_0,\varepsilon _0),$ rencontre $U_0$ par densité en un point $x_1.$ Soit $\varepsilon _1$ tq $\varepsilon _1\le \varepsilon _0 /2$ et $B'(x_1,\varepsilon _1)\subset U_0\cap B(x_0,\varepsilon _0)$ qui est ouvert.\\
    On construit alors par récurrence $x_{n+1}\in B(x_n,\varepsilon _n)\cap U_n$ vérifiant $\varepsilon _{n+1}\le \varepsilon _n /2$ et $B'(x_{n+1},\varepsilon _{n+1})\subset U_n\cap B(x_{n},\varepsilon _n)$. Or $B'(x_{n+1},\varepsilon _{n+1})$ suite de fermés emboités de diamètre $\le 2\varepsilon _n\to 0$.\\
    Soit $x_*\in \bigcap\limits_{n\in \mathbb{N} } B'(x_n,\varepsilon _n)$ par th des fermés emboités, alors $\forall n\in \mathbb{N} ,\ x_*\in B'(x_{n+1},\varepsilon _{n+1})\subset U_n $. Donc on a bien la densité de $\cap U_n$.
\end{proof}
\begin{ex}
    Soit $(q_k)$ une énumération de $\mathcal{O}$ posons $U_x:=\cup ]q_k-\frac{1}{nk^2},q_k+\frac{1}{nk^2}[$ Alors $Leb(U_n)\le \sum\limits_{k\ge 1}^{} \frac{2}{nk^2}=\frac{\pi^2}{3n}$. Ainsi $\bigcap U_n$ est une intersection d'ouverts denses mais de mesure nulle.
\end{ex}

\begin{corollaire}
    Soit $(\gamma,d)$ un espace métrique complet et $(F_n)$ une suite de fermé d'intérieur vide. Alors $\bigcup\limits_{n\in \mathbb{N} } F_n$ est d'intérieur vide.
\end{corollaire}

Terminologie de Baire
\begin{itemize}
    \item Une intersection dénombrable d'ouverts est un $G_{\delta}$
    \item Une union dénombrable de germés est un $F_\sigma$
    \item Un ensemble qui contient un $G_{\delta}$ dense est dit gras
    \item Un ensemble contenu dans un $F_\sigma$ d'intérieur vide est dit maigre
\end{itemize}

\begin{remarque}
    Soit $(X,d)$ un espace métrique complet et sans points isolés. Alors tout ensemble $A$ gras est indénombrable.
\end{remarque}
\begin{proof}
    Soit $x\in X$. Alors $\{x\} $ est fermé (car $x$ n'est pas un point isolé) et d'intérieur vide. Donc $X\backslash \{x\} $ est un ouvert dense. Si par l'absurde $A$ est dénombrable, alors $A\cap \left( \bigcap\limits_{x\in A} X\backslash \{x\}  \right) $ contient une intersection dénombrable d'ouvert denses donc est dense par Baire. Contradiction !
\end{proof}

\subsection{Applications de Baire aux opérateurs linéaires continus.}
\begin{theoreme}[Banach-Steinhaus]
    Soit $E$ un Banach, $F$ un evn et $A\subset L_c(E,F)$ un ensemble d'applications linaires continues. Si $A$ est simplement borné (ie $\forall x\in E,\ \sup_{u\in A}\|u(x)\|_F<\infty ) $ alors $A$ est uniformément borné (ie $\sup_{u\in A}\vertiii{u}<\infty $, ie on peut choisir $C(x):=\sup_{u\in A}\vertiii{u}\|x\|_E$) .
\end{theoreme}
\begin{proof}
    (via Baire) Pour tout $k\in \mathbb{N} ,$ posons $E_k:=\{x\in E\ |\ \forall u\in A,\ \|u(x)\|_F\le k \} $. C'est un fermé, comme intersection de fermés.
    Par hypothèse, $\bigcup\limits_{k\in \mathbb{N} } E_k=\underbrace{E}_{\mathclap{\text{intérieur non vide}}}$, car $x\in E_k$ dès que $k\ge C(x)$.
    Donc par Baire, l'un au moins des $E_k$ est d'intérieur non vide. 
    Disons $B(x,r)\subset E_k,$ pour un certain $x\in E,k\in \mathbb{N}$. 
    Par symétrie, $B(-x,r)\subset E_k$. 
    Par continuité, $B(0,r)\subset E$ (car $\|u(k)\|\le \frac{\|u(x+k)\|+\|u(-x+k)\|}{2}$). 
    On en déduit $\forall y\in B(0,x),\ \forall u\in A,\ \|u(y)\|\le k$.
    Donc $\forall u\in A,\ \vertiii{u}\le \frac{k}{r}$.
    Comme annoncé.
\end{proof}
\begin{corollaire}
    Soit $E,F$ des Banach et $u_n\in L_c(E,F).$ On suppose $u_n(x)\xrightarrow[n\to +\infty]{} u(x)$ pour tout $x\in E.$ Alors $u$ est linéaire continue. (ie Une limite simple de fonctions linéaire continues est linéaire continue.)
\end{corollaire}
\begin{proof}
    La linéarité de $u$ découle de la limite simple : $u(\lambda x+y)=\lim u_n(\lambda x+y)=\lim \lambda u_n(x)+u_n(y)=\lambda u(x)+u(y)$.\\
    La suite $(u_n)$ est simplement bornée, en effet $\forall x\in E,\ (u_n(x)) $ est convergente donc bornée. Par Banach-Steinhaus $\vertiii{u_n}\le C_*\|x\|.$ Donc $\|u(x)\|=\lim_{n\to \infty }\underbrace{\|u_n(x)\|}_{\mathclap{\substack{\le \vertiii{u_n}\|x\|}}}\le C_*\|x\|$ donc $u$ est continue.
\end{proof}
\begin{corollaire}
    Soit $E$ un Banach et $A\subset E^*$ simplement borné (ie $\forall x\in E,\ |l(x)|_{l\in A} $ est borné) "faiblement borné". Alors $A$ est uniformément borné (ie $(\|l\|_{E^*})$ est borné)
\end{corollaire}

\begin{proof}
    Prendre $F=\mathbb{K}$ le corps de base $(\mathbb{K}=\mathbb{R} $ ou $\mathbb{C}$) et appliquer Banach-Steinhaus
\end{proof}
\begin{remarque}
    Il y a une version duale de ce résultat mais les preuves nécessitent le théorème de Hahn-Banach
\end{remarque}
\begin{ex}
    Il existe $f\in C^ (\Pi,\mathbb{C})$ donc la série de Fourier diverge en 0. $\Pi:=\mathbb{R} /_{2\pi\mathbb{Z} }=[0,2\pi[$. On a $L_N(f):=\frac{1}{2\pi}\sum\limits_{|n|\le N}^{} \int_0^{2\pi}f(t)e^{-int}dt$, alors $\exists f\in C^0,\ \exists \varphi $ extractrice telle que $|L_{\varphi (n)}(f)|\to \infty $.
\end{ex}
\begin{proof}
    On a
     \begin{align*}
         L_N(f)&=\int_0^{2\pi}f(t)\underbrace{\sum\limits_{|n|\le N}^{} e^{-i nt}}_{\mathclap{D_n(t)}}dt\\
         D_n(t)&=e^{-i Nt}\frac{1-e^{i(2N+1)t}}{1-e^{it}}\\
               &=\frac{\sin \left( \frac{1}{2}(2N+1)t \right) }{\sin\left( \frac{1}{2}t \right) }
     \end{align*}.
     On munit $C^0$ de $\|\|_\infty $ qui en fait un complet. Donc $\vertiii{D_n}=\sup_{\|f\|_\infty \le 1}\int_0^{2\pi}f(t)D_n(t)dt=\int_0^{2\pi}|D_n(t)|dt$ en appliquant le sugne de $D_n$.\\
      \begin{align*}
          \vertiii{L_N}&=\int_{-\pi}^\pi\frac{/|\sin\left( \frac{1}{2}(2N+1)t \right) }{|\sin \frac{t}{2}|}dt\\
                       &\ge \int_{-\pi}^\pi |\sin\left((2N+1)\frac{t}{2}\right)|\frac{dt}{t} &\text{car $|\sin t|\le |t|$ }\\
                       &= 2\int_0^{2N+1)\frac{\pi}{2}}|\sin s|\frac{ds}{s} &\text{par symétrie}
     \end{align*}.
     Diverge car $\int _0^\infty \frac{|\sin s|}{s}ds=\infty .$ (découper l'intégrale selon $\bigcup\limits_{k\in \mathbb{N} } [k\pi,(k+1)\pi[.$ \\
     Ainsi $(\vertiii{L_n}$ est bien bornée. Donc par contraposée de Banach-Steinhaus $\exists f\in E=(C^0(\Pi,\mathbb{C}),\|.\|_\infty ),\ \sup_{n\in \mathbb{N} }|L_n(f)|=\infty .$
\end{proof}

\begin{theoreme}[Banach-Steinhaus dans les Fréchets]
    Soit $(E,(|.|_n))$ et $(F,(|.|'_n))$ des Fréchets et $A\subset L(E,F)$ une famille d'applications linéaires continues. Si $A$ est simplement borné, i.e. $\forall x\in E,\ \forall m\in \mathbb{N} ,\ \sup_{u\in A}|u(x)|_m<\infty . $ Alors $A$ est équicontinue ie $\exists w,$ module de continuité $\forall x,y\in E,\ d_F(u(x),u(y))\le w(d_E(x,y)). $
\end{theoreme}
\begin{proof}
    Soit $m\in \mathbb{N} $ fixé. Posons $E_k:=\{x\in E\ |\ \forall u\in A,\ |u(x)|'_m\le k \}  $. Alors $E_k$ est fermé et comme avant on a l'union fait l'ensemble non vide. Par Baire il y a un $E_k$ non d'intérieur vide. Par symétrie et continuité il continent un voisinage de 0. Donc $\exists N(m), r>0,\ \{x\in E\ |\ \forall n\le N(m),\ |x|_n<r \} \subset E_k.$ On en déduit $|u(x)|_m'\le \frac{k}{r}\max_{n\le N(m)}|x|_m.$ Noter que $\frac{k}{r}$ et $N(m)$ sont indépendant de $u\in A.$ On en déduit l'équicontinuité en 0 puis en tout point par linéarité. Rappelons $d_F(x,y)=\max_{m\in \mathbb{N} }\min(2^{-m},|x-y|_m')$ et $d_E(x,y)=\max_{m\in \mathbb{N} }\min(2^{-m},|x-y|_m)$.
\end{proof}
\begin{theoreme}[Application ouverte, Banach]
    Soit $E,F$ Banach et $u\in L(_cE,F)$ surjective. Alors $u$ est ouverte, ie $\underbrace{u(O) \text{est un ouvert}}_{\mathclap{\text{1}}}$ dans $F$ pour tout ouvert $O$ de $E$.\\
    Ou, de manière équivalente :
    \begin{enumerate} \setcounter{enumi}{1}
        \item $\exists C,\ \forall y\in F,\ \exists x\in E,\ y=u(x) $ et $\|x\|\le C\|y\|$
        \item $\exists r>0,\ B_F(0,r)\subset u(B_E(0,1))$.
    \end{enumerate}
\end{theoreme}
\begin{proof}.
\begin{itemize}
    \item[$1\Rightarrow 3$] $u(B_E(0,1))$ est un ouvert car image d'un ouvert et continent 0 donc continent un voisinage de 0 dans $F$.
    \item[$3\Rightarrow 1$] Soit $U$ ouvert de $E$ et $x\in U.$ Soit $\varepsilon >0$ tq $B_E(x,\varepsilon )\subset U.$ Alors 
    \begin{align*}
        u(U) &\supset u(B_E(x,\varepsilon ))\\
        &=u(x)+\varepsilon u(B_E(0,1))\\
        &\supset u(x)+\varepsilon B_F(0,r)\\
        &=B_F(u(x),\varepsilon r)
    \end{align*}.
    \item[$3\Rightarrow 2$] Soit $y\in E\backslash \{0\} ,$ alors $\frac{y}{\|y\|}\frac{r}{2}\in B(0,r).$ Donc $\frac{y}{\|y\|}\frac{r}{2}=u(x_*)$ pour un $x_*\in B(0,1).$ Donc $y=u(\underbrace{\frac{2}{r}\|y\|x_*}_{\mathclap{x}})$ et $\|x\|\le \frac{2}{r}\|y\|.$
    \item[$2\Rightarrow 3$] Soit $r=\frac{1}{C},$ si $y\in B(0,r),$ alors $\exists x\in B(0,1),\ y=u(x).$
\end{itemize}
Preuve du point 2 à partir des hypothèses. Par surjectivité,
${\bigcup\limits_{n\in \mathbb{N}} \overline{u(B_E(0,n))}=F.}$ Par Baire, $\exists n,\ \overline{u(B_E(0,n))}$ est d'intérieur non vide. Par symétrie et continuité, $\exists r>0,\ B_F(0,r)\subset \overline{u(B_E(0,n))}.$ Donc $\forall y\in B_F(0,r),\ \forall \varepsilon >0,\ \exists x\in B_E(0,n),\ \|y-u(x)\|<\varepsilon .$ Par homogénéité $\forall y\in  F,\ \forall \varepsilon >0,\ \exists x\in E,\ \|y-u(x)\|<\varepsilon $ et $\|x\|_E\le C\|y\|_F$ pour $C=\frac{2n}{r}$.\\
Soit $y_0\in F\backslash \{0\} $ dont on veut construire un antécédent. On choisit $x_0\in E$ tq $\|x_0\|\le C\|y_0\|$ et $\|y_0-u(x_0)\|\le \frac{\|y_0\|}{2}.$ On pose $y_1=y_0-u(x_0).$ OPS $y_1\neq 0$ sinon on a bien un antécédent. Par récurrence on construit $(y_n),(x_n)$ tq $\|x_n\|\le C\|y_n\|$ et $\|y_n-u(x_n)\|\le \|y_n\|/2.$ On a $y_{n+1}=y_n-u(x_n).$ Alors $\|y_n\|\le 2^{-n}\|y_0\|$ par récurrence et $\|x_n\|\le C_2^{-n}\|y_0\|.$ Or $\sum\limits_{n=0}^{N} x_n\to x_*$ par complétude de $E.$ Par ailleur
\begin{align*}
    y_n&=y_{n-1}-u(x_{n-1})\\
       &=y_{n-2}-u(x_{n-1}+x_{n-2})\\
       &\cdots\\
       &=y_0-u(\sum\limits_{k<n}^{} x_k).
\end{align*}
Donc $\|y_0-u(\sum\limits_{k<n}^{} x_k)\|=\|y_n\|\to 0$ et $\to \|y_0-u(x_*)\|$. On en conclut $y_0=u(x_*)$ et $x_*\le \sum\limits_{n=0}^{\infty} \|x_n\|\le \sum\limits_{n=0}^{\infty} C_2^{-n}\|y_0\|\le 2C\|y_0\|.$
\end{proof}

\begin{corollaire}[Isomorphisme de Banach]
    Si $E,F$ est de Banach et $u\in L_c(E,F)$ bijective, alors $u^{-1} \in L_c(F,E)$
\end{corollaire}
\begin{proof}
    $u^{-1} $ est linéaire comme inverse d'une application linéaire. Montrons qu'elle est continue. Si $U\subset E$ est ouvert alors $(u^{-1})^{-1} (U)=u(U) $ est ouvert par th de l'application ouverte, ce qui conclut ($u$ est bijective donc surjective).
\end{proof}
\begin{corollaire}
    Soit $E$ un espace vectoriel muni de $\|.\|$ et $\|.\|'$ tq $(E,\|.\|)$ et $(E,\|.\|')$ sont complets. Supposons $\exists C,\ \forall x\in E,\ \|x\|'\le C\|x\|. $ Alors $\exists c>0,\ \forall x\in E,\ \|x\|'\ge  c\|x\| $ (équivalence des normes).
\end{corollaire}
\begin{proof}
    L'application $Id:(E,\|.\|)\to (E,\|.\|')$ est continue car $\|Id(x)\|'=\|x\|'\le C\|x\|$ pour tout $x\in E$ et bijective. Par le corollaire isomorphisme de Banach, $Id^{-1} : (E,\|.\|')\to (E,\|.\|)$ est continue ie $\|Id^{-1} (x)\|=\|x\|\le \tilde{C}\|x\|'.$ On pose $c=\frac{1}{\tilde{C}}$.
\end{proof}
\begin{theoreme}[Graphe fermé]
    Soit $E,F$ de Banach et $u:E\to F$ linéaire. Sont équivalent :
    \begin{itemize}
        \item $u$ est continue
        \item $\mathcal{G}(u):=\{(x,u(x))\in E\times F\ |\ x\in E\} $ est fermé
    \end{itemize}
\end{theoreme}
\begin{proof}
    On rappel que $E\times F$ est un Banach pour la norme $\|(x,f)\|_{E\times F}:=\|x\|_E+\|y\|_F.$
    \begin{itemize}
        \item[$1\Rightarrow 2$] $\mathcal{G}(u)=\{(x,y)\in E\times F\ |\ y-u(x)=0\} $. Or $\varphi  :\begin{aligned}
            E\times F &\longrightarrow F \\
            (x,y) &\longmapsto y-u(x)
        \end{aligned}$ est continue donc $\mathcal{G}(u)=\varphi ^{-1} (\{O_F)\} $ est fermé.
    \item[$2\Rightarrow 1$] $\mathcal{G}(u)$ est un sous espace vectoriel fermé de $E\times F$ donc c'est un Banach pour la norme $\|.\|_{E\times F}.$ De plus l'application $\varphi : \begin{aligned}
         \mathcal{G}(u)\times F&\longrightarrow E \\
         (x,u(x))&\longmapsto x
    \end{aligned}$ est linéaire, continue et bijective. (on aurait aussi pu faire avec équivalence des normes) Par l'isomorphisme de Banach, $\varphi ^{-1} $ est continue. Donc $\|x\|+\|u(x)\|=\|\varphi ^{-1}  (x)\|\le C\|x\|.$ Finalement $\|u(x)\|\le (C+1)\|x\|.$
    \end{itemize}

\end{proof}
