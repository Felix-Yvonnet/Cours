

\section{Analyse}
\subsection{ Rappel de topologie }
        Un espace topologique est une paire $(X, \mathbb{U}$), où $X$ est un ensemble et $\mathbb{U}\subset P(X)$ est l'ensemble des ouverts satisfait:\\
        \begin{enumerate}
            \item $\emptyset \ X\in \mathbb{U}$
            \item $\forall \mathcal{U}\subset \mathbb{U}\ \bigcup_{u\in \mathcal{U}} u\in \mathbb{U}$
            \item $\forall u,v\in \mathbb{U}\ u \cap v \in \mathbb{U}$
            \end{enumerate}
\begin{remarque}
si $\mathcal{U}=\emptyset $ alors $\bigcap_{u\in \mathcal{U}} u = \emptyset $. En revanche l'intersection vide n'est pas définie
\end{remarque}
\begin{remarque}
Un fermé est le complémentaire d'un ouvert. $\emptyset \ X$ sont fermés et les fermés sont stable par union finie et intersection quelconque
\end{remarque}

\begin{definition}
     Soit $A\subset X$ où $(X,\mathbb{U})$ esp topo. On définit l'intérieur $\overset{\circ}{A}:=\bigcap_{\text{F fermé\\$A\subset F$}}F$
\end{definition}
On note que $X\backslash\mathring{A}=\overline{X\backslash A}$ et $X\backslash\overline{A} =\overbrace{X\backslash A}^{\circ}$

\subsection{Comparaison de topologies :}
\begin{definition}
    Soit $X$ un ensemble muni des toppo $\mathbb{U},\mathbb{V}$. On dit que $\mathbb{U}$ est plus fine que $\mathbb{V}$ si $\mathbb{U}\supset\mathbb{V}$
\end{definition}
\begin{ex}
la topo "finie" est définie par $\mathbb{U}=P(X)$, "grossière" par $\mathbb{U}=\{\emptyset, X\}$
\end{ex}

\begin{definition}
     $X$ ens et $\mathbb{U}_0\subset P(x)$. La topo $\mathbb{U}$ engendrée par $\mathbb{U}_0$ est la plus grossière contenant $\mathbb{U}_0$. $\mathbb{U}=\cap \{\mathbb{U}'\subset P(X)|\mathbb{U}'\text{ topo et }\mathbb{U}'\supset \mathbb{U}_0\}$. Bien une topo car intersect.
\end{definition}
\begin{remarque}
 Les éléments de $\mathbb{U}$ sont $X$ et les unions qcq d'intersections finies d'éléments de $\mathbb{U}_0$. $u\in \mathbb{U}_0 \Leftrightarrow u=X \vee u=\cup \cap u_{ij}$
\end{remarque}
\begin{definition}
 Une base d'ouverts sur un ens $X$ est une partie $w_0\subset P(X)$ tq (couverture) $\bigcup_{u\in \mathbb{U}_0}u=X$ et (stabilité par intersect) $\forall u,v\in \mathbb{U}_0, \ \forall x\in u\cap v\ \exists w\in \mathbb{U}_0\ x\in w\text{ et } w\subset u\cap v$
\end{definition}
\begin{proposition}
    La topo $\mathbb{U}$ engendrée est une base d'ouvert $\mathbb{U}_0$ est constituée des unions qcq de $\mathbb{U}_0$
\end{proposition}
\begin{proof}
    On note que $X=\bigcup_{u\in \mathbb{U}_0}u$ est bien une union \ldots\\
    Si $u,v\in \mathbb{U}_0$, on note $W_x\in \mathbb{U}_0$ tq $x\in W_x$  et $W_x \subset u\cap v$ pour tout $x\in u\cap v$. Alors $u\cap v=\bigcup_{x\in u\cap v}W_x$, puis $\bigcup_{i\in I}\underbrace{\bigcap_{1\le j\le J(i)}u_{ij}}_{\text{S'écrivent comme réunion de la base d'ouverts}}$
\end{proof}
\begin{ex}
    (topo de l'ordre) : Soit $(X,\le )$ un ensemble totalement ordonné abec au moins 2 élems. On définit une base d'ouverts par les intervalles : $]-\infty , b[, ]a,b[, ]a,\infty [, a,b\in X$
\end{ex}
\begin{proof}
    Si $a<b\in X$ alors $X=]-\infty, b[\cup]a, \infty [$. De plus De plus $] \alpha, \beta[\cap ] \delta, \gamma[ = ]\min, \max[$
\end{proof}
\begin{ex}
     (topo produit) : $(X_i,\mathbb{U}_i)_{i\in I}$ une famille d'espace topos, on def la topo prod par la base d'ouverts $\{\pi_{i\in I}u_i|\forall i\in I, u_i\in \mathbb{U}_i \text{ et }u_i=X_i \text{ sauf pour un nombre fini de } i\in I \}$
\end{ex}
\begin{ex}
    Si $X_i=X, \forall i\in I$, alors $\pi_{i\in I}X=X^I$ est l'ensemble des fonctions de $I$ dans $X$. La topo produit sur $X^I$ correspond à la convergence simple. $f_n\to _{n\to \infty }f\Leftrightarrow \forall i\in I,\ f_n(i)\to f(i)$
\end{ex}

\subsection{ Voisinages :}
\begin{definition}
    $(X,\mathbb{U})$ un espace topo et $x\in X$. Un voisinage $V$ de $x$ est une partie $V\subset X$ tq $\exists u\in \mathbb{U},\ x\in u \wedge u\subset V$. De manière équivalente $V$ vois de $x$ $\Leftrightarrow $ $x\in \mathring{V}$.\\
     On note $\mathcal{V}_x$ l'ensemble des voisinages de $x\in X$.
\end{definition}

\begin{definition}[Caractérisation de l'adhérence]
$\forall A\subset X,\ \overline{A}=\{x\in X|\forall v\in \mathcal{V}_x,\ A\cap v\neq \emptyset  \}$ aussi $\mathring{A}=\{x\in X|\exists v\in \mathcal{V}_x,\ v\subset A\}$
\end{definition}
\begin{definition}
une partie $W_x\subset V_x$ est une base de voisinage ssi $\forall v\in V_x,\ \exists w\in W_x,\ w\subset v$
\end{definition}
\begin{definition}
     une topo $\mathbb{U}$ de $X$ est :
\begin{enumerate}
    \item A \underline{base dénombrable de voisinages} ssi tout point $x\in X$ admet une base dénombrable $W_x$ de voisinage.
    \item A \underline{base dénombrable} si elle est engendrée par une base d'ouverts dénombrable.
\end{enumerate}
\end{definition}
\begin{remarque}
    si $(X,d)$ est un espace métrique et $x\in X$, alors $W_x=\{B(x,\frac{1}{n}| n\in \mathbb{N} ^*\}$ est une base de vois
\end{remarque}
\begin{remarque}
    Si $(X,d)$ est un espace métrique admettant une suite $(x_n)_{n\in \mathbb{N} }$ dense, alors une base dénombrable d'ouverts est $\mathbb{U}_0=\{B(x_{n}, r)|n\in \mathbb{N} \ r\in \mathbb{Q} \}$
\end{remarque}
\begin{proof}
    $\mathbb{U}_0$ recouvre bien $X$. Soit $x\in B(x_{n}, r)\cap B(x_{n}, s)=BB$ et $\varepsilon >0 \in \mathbb{Q} $ tq $B(x,\varepsilon )\subset BB$. Soit $k\in \mathbb{N} $ tq $x_k\in B(x,\varepsilon /2)$. Alors $x\in B(x_k, \varepsilon /2)\subset B(x,\varepsilon /2 + \varepsilon /2)=B(x,\varepsilon)$. \\

Par le même raisonnement, $\mathbb{U}$ contient les vois arbitrairement petits de tt pt. Donc c'est une base d'ouverts pour les topos de $X$.
\end{proof}
\begin{proposition}
    Soit ($X,\mathbb{U}$) à base dénombrable de voisinage. Alors $\forall A\subset X,\ \overline{A}=\{x\in X|\exists (x_{n})_{n\in \mathbb{N} }\in A^\mathbb{N},\ x_{n}\to x\} $.
\end{proposition}
\begin{proof}
$(V_n)_{n\in \mathbb{N} }$ une base de voisinages de $x$, soit $x_{n}\in \underbrace{V_0\cap \cdots\cap V_n\cap A}_{\text{une $\cap $ finie de vois de $x$\\est un vois de $x$}},\ \forall n \in \mathbb{N}$ Alors $x_{n}\to x$.($\Leftrightarrow \forall v\in V_x,\ \exists N, \forall n\ge N, \ x_{n}\in V$).
\end{proof}
\begin{proposition}
    Prop : soit $(X,\mathbb{U})$ esp topo à base dénombrable de voisinage et $(x_{n})_{n\in \mathbb{N} }\in X^\mathbb{N} $. Alors toutes valeurs d'adhérence de $(x_{n})$ est la limite d'une sous suite.
\end{proposition}
On rappelle que $Adh((x_{n}))= \bigcap_{N\in \mathbb{N} }\Bar{\cup \{x_{n}|n\ge N\}}$ .
\begin{proof}
     on note que $Adh(x_{n})=\{x\in X|\forall v\in V_x,\ \{n\in \mathbb{N} |x_{n}\in V\} \text{ est in²fini} \}$. La preuve suite comme précédemment en choisissant $(V_n)$ base de vois $\searrow$ pour l'inclusion et $x_{\sigma(n)}\in V_x$ avec $\sigma$ strict $\nearrow$.
\end{proof}

\subsection{ Séparation :}
\begin{definition}
Un espace topo est séparé ssi $\forall x,y\in X,\ x\neq y\Rightarrow \exists u,v\in \mathbb{U},\ x\in u, y\in v, u\cap v=\emptyset $.\\
Si $(X,\mathbb{U})$ est séparé, alors toute suite a au plus une limite (Haussdorff, $T_2$.\\

\end{definition}
\begin{definition}
	Un espace $(X,\mathbb{U})$ satisfait l'axiome $T_1$ de Kolmogorov, ssi $\forall x\neq y\in X\ \exists u\in \mathbb{U},\ x\in u \wedge y \not \in u$.\\

\end{definition}
\begin{ex}
 (topo $T_1$ mais pas $T_2$) :
\begin{enumerate}
	
   \item $\mathbb{N} $ muni de topo cofinie : les fermés sont les ensembles finis
\item $\mathbb{C}^d$ muni de la topo de Zariski : les fermés ont les ensembles algébriques $F=\{x\in \mathbb{C}^*|P_1(x)=\cdots P_n(x)=0\}\ n\ge 0;\ P_1,\cdots,P_n\in \mathbb{C}[X]$
\end{enumerate}
\end{ex} 
\begin{ex}
	
La suite $(n)_{n\in \mathbb{N} }$ converge vers tous les points de $\mathbb{N} $ pour la topo cofinie. En effet, soit $k\in \mathbb{N} $ et $V$ un voisinage de $k$. Alors $V$ contient tous les points sauf un nombre fini. Donc tous les termes de la suite à partir d'un certain rang.\\
De même, une suite de point qui n'est continue dans aucun ensemble algébrique propre converge vers tt point de $\mathbb{C}^d$ pour Zariski.
\end{ex}

\subsection{ Continuité :}
\begin{definition}
	Soit $(X,\mathbb{U})$ esp topo. Une application $f:X\to Y$ est continue en $x\in X$ ssi $\forall W\in V_{f(x)}, \ f^{-1}(W)\in V_x$. (ie $\forall W\in V_{f(x)},\ \exists V\in V_x,\ f(V)\subset W$). $f$ est continue $\overset{def}{\Leftrightarrow }$ pour tout $x\in X$, $f$ est continue en $x$. \\

\end{definition}
\begin{definition}
	 $(X,\mathbb{U}), (Y,\mathbb{V})$ esp topos et $f:X\to Y$. Sont équivalents :
    \begin{enumerate}
        \item $f$ continue
        \item $\forall V\in \mathbb{V}\ f^{-1}(V)\in \mathbb{U}$ (l'image réciproque d'un ouvert est un ouvert)
        \item $\forall F\text{ fermé de }Y,\ f^{-1}(F) \text{ fermé de }X$. (recip fermé est fermé)
        \item $\forall A\subset X,\ f(\overline{A})\subset \overline{f(A)}$
    \end{enumerate}
Une composition de fcts continue est continue, l'image par une fonction continue d'une suite convergente est convergente.\\

\end{definition}
\begin{ex}
	 $X$ un ensemble et $(f_i:X\to Y_i)$ une famille d'applications vers des espaces topos. On peut considérer la topo la moins fine qui les rend continue. Elle est engendrée par les $\{f^{-1}(U_i)|i\in I, U_i\in \mathbb{U}_i\}$.

\end{ex}

\subsection{Espace métrique}
\begin{definition}
    $(X,d)$ espace métrique où $d:X\times X\to \mathbb{R} $est application distance, ssi elle satisfait :
    \begin{enumerate}
        \item(Séparation) $\forall x,y\in X,\ d(x,y)\ge 0$ (et $d(x,y)=0\Leftrightarrow x=y$).
        \item (Symétrie) $\forall Ax,y\in X,\ d(x,y)=d(y,x)$
        \item (Inégalité triangulaire) $\forall x,y,z\in X,\ d(x,z)\le d(x,y)+d(y,z)$
    \end{enumerate}
\end{definition}

\begin{definition}
    $\forall x\in X,\ \forall r>0$ on définit :
    \begin{itemize}
        \item $B(x,r):=\{y\in X|d(x,y)<r\} $
        \item $B^f(x,r):=\{y\in X|d(x,y)\le r\} $
    \end{itemize}
\end{definition}

Les topologies associées à un espace métrique est celle induite par la base d'ouverts $\{B(x,r)|x\in X,r>0\} $.

\begin{definition}
    $(X,\mathbb{U})$ est séparable $\Leftrightarrow \exists A\subset X $ dénombrable $\overline{A}=X$\\
$(X,\mathbb{U})$ est séparé $\Leftrightarrow $ il satisfait l'axiome $T_2$.
\end{definition}

On peut utiliser dans un espace métrique les caractérisations séquentielles de l'adhérence et sur les fonctions continues.

\begin{definition}
    Un module de continuité est une aplication $x:\mathbb{R} ^+\to [0,\infty ]$, tq $w(x)\to_{x\to 0} 0$
\end{definition}

Soit $(X,d_x)$ et $(Y,d_y)$ des espaces métriques et $f:X\to Y$ est :
\begin{itemize}
    \item continue en $x\in X$ ssi il existe $w_x$ module de continuité tq $d_y(f(x),f(y))\le w_x(d(x,y)), \forall y\in X$.
    \item uniformément continue ssi il existe $w$ un module de continuité tq $d_y(f(x),f(y))\le w(d(x,y)) \forall x,y\in X$.
    \item Lipschitzienne ssi $\exists C[w=CId],\ \forall x,y, d(f(x),f(y))\le Cd_x(x,y)$.
    \item $ \alpha$-Holderienne[$ 0<\alpha<1$]  ssi $\exists C[w=Cr^2],\ \forall x,y,\ d_y(f(x),f(y))\le Cd_x(x,y)^2$.
\end{itemize}

\begin{remarque}
    Si $w$ est un module de continuité,
    \begin{itemize}
        \item $\tilde{w}(r):=\sup_{0\le s\le r} w(s)$ est \ldots croissant et $\tilde w\ge w$
        \item $\hat{w}(r);=\frac{1}{2} \int_0^{2r}\tilde{w}(s)ds$ est \ldots croissant et continue
        \item $\hat{w}(r)\ge \tilde{w}(r)\ge w(r)$.
    \end{itemize}
\end{remarque}

\subsection{Espaces vectoriels normalisés (evn)}
 $\mathbb{K}=\mathbb{R}$ ou $\mathbb{C}$
\begin{definition}
    une evn est une paire ($E,\|\|)$ où $E$ est un $\mathbb{K}$ espace vectoriel et $\|\|$ est une norme sur $E$. $\|.\|$ satisfait :
    \begin{itemize}
        \item (Séparation) $\forall x\in E,\ \|x\|\ge 0$
        \item (Homogénéité) $\forall c\in E,\ \forall  \lambda\in \mathbb{K},\ \|\lambda x\|=\|\lambda\|\|x\|$
        \item (Inequality triangulaire) $\forall x,y \in E,\ \|x+y\|\le \|x\|+\|y\|$
    \end{itemize}
\end{definition}

On lui associe $d(x,y)=\|x-y\|$ est la topologie associée.

\begin{propriete}
    Soit $E,F$ des evn, une application linéaire $u:E\to F$ est continue ssi $\exists C,\ \forall x\in E,\ \|u(x)\|_F\le C\|x\|_E$ ie $u$ est continue et linaire alors elle est lipschitzienne.
\end{propriete}

On note $L(E,F)$ l'espace vectoriel des applications \underline{linéatire et continues} de $E$ dans $F$.

C'est un evn pour la norme $\mid ||u\|\| |_{L(E,F)}:=\sup\{\|u(x)\|_F|x\in E, \|x\|_E\le 1\} $. \\
En particulier $E^*=L(E,\mathbb{K})$ l'espace vectoriel des formes linéaires continues est aussi un evn.

\begin{ex}
    Soit $(X,d)$ un espace métrique, alors $C_b(X,\mathbb{K})$ est un evn pour la norme $\|f\|_\infty :=\sup\|f(x)\|$.\\
    De même, pour $ 0<\alpha <1$ $C^\alpha_b(x)$ est un evn muni de la norme $\|f\|_{C^\alpha }:=\|f\|_\infty +\|f\|_{C^\alpha }$ où $\|f\|_{C^\alpha }:=\sup \frac{\|f(x)-f(y)}{d(x,y)^\alpha }\|$.
\end{ex}
De même les fonctions Lipschitziennes.

\begin{ex}
    Soit $\Omega\subset \mathbb{R} ^d$ ouvert et $n\in \mathbb{N} $. $C^\alpha _b(\Omega)$[underscore b pour bornée] est un evn pour la norme \ldots\\
    $C^n_b(\overline{\Omega})$ muni de la même norme est constitué des $f\in C^n_b(\Omega)$ tq $\partial_\alpha f$ s'étend continuellement à $\Bar{\Omega}$. [Rem : on peut montrer qu'elles admettent une extension continue a une voisinage de $x$].
\end{ex}

Si $(X,\mu)$ est un espace mesuré, on note $L^*(X,\mu):=\{f:X\to \mathbb{R} |f \text{ mesurable}\} /$\~{} où $f~g\Leftrightarrow f=g\ \mu$-presque partout.\\

On définit $\|f\|_p:=\left( \int\|f\|^p \right)^{\frac{1}{p}}$ où $p\in [1,\infty [$. On a les evn $L^p$ muni de $\|.\|_p$.\\
\begin{proof}
    L'homogénéité, la séparation et la mutabilité sont clairs. L'inégalité triangulaire est appelée inégalité de Minkowski :\\
    Soit $p\in [1,\infty [$, $f,g\in L^p(X,\mathbb{K})$ OPS $\|f\|_p>0,\|g\|_p>0, \|f\|_p+\|g\|_p=1$. Posons $F=\frac{f}{\|f\|_p}$ et $G=\frac{f}{\|g\|_p}$. \\
    Alors $\|f(x)+g(x)\|^p=\|(1-\lambda)F(x)+\lambda G(x)\|$ pour $\lambda=\|g\|_p$. Le module est convexe et la fonction puissance est aussi convexe donc la composition l'est. Donc $\|f(x)+g(x)\|\le (1-\lambda)\|F(x)\|^p+\lambda\|G(x)\|^p$. Donc tout va bien la suite en exercice :)
\end{proof}


\subsection{Espaces vectoriels topologiques localement convexes (ev+lc)}

\begin{definition}
    Un ev+lc est un $\mathbb{K}$-ev $E$ muni d'une famille de semi normes ($\|.\|_i)_{i\in I}$ (pas la séparation). La topo associée est définie par la base d'ouverts de la forme $\{y\in E| \forall i\in I_0, \|x-y\|_i<\varepsilon \} $ avec $x\in E, \varepsilon >0$ et $I_0\subset I$ fini.
\end{definition}

\begin{remarque}
    Une semi norme est une application $\|.\|:E\to \mathbb{R} $ positive et homogène, satisfaisant l'inégalité triangulaire. (pas de séparation).
\end{remarque}

\begin{remarque}
    La topo n'est pas automatiquement séparée, cela doit lyre vérifié. Tout evn est un ev+lc avec une famille $(\|.\|_i)$ réduite à l'élément $\|.\|$.
\end{remarque}

\begin{proposition}
   une application linéaire $u:E\to E$, avec $E, (\|.\|_i)$ et $F,(\|.\|_j)$ est continue ssi $\forall j\in J, \exists I_0\subset I,\ \exists C,\ \forall x\in E\ \|u(x)\|_j\le C \sum\limits_{i\in I_0}^{} \|x\|_i$. \\
\end{proposition}

En particulier une forme linéaire $u:E\to \mathbb{K} $ est continue ssi $\exists I_0\subset I$ fini, $\exists C,\ \forall x\in E\ \|u(x)\|\le C \sum\limits_{i\in I_0}^{} \|x\|_i$.

\begin{proof}
   Supp $u$ continue, soit $j\in J$, on a le voisinage de 0: $W:=\{y\in E|\|y\|_j<1\} $. On a $u(0)=0$ par linéarité. Par continuité, il existe un voisinage $V$ de 0 dans $E$ tel que $u(V)\subset W$. $V$ contient un élément de la base de voisinage donc $\exists \varepsilon >0,\ \exists I_0\subset I$ fini, $\{x\in E|\forall i\in I_0,\ \|x\|_i<\varepsilon  \} \subset V$.\\
   On a montré que : $(\forall i\in I_0,\ \|x\|_i<\varepsilon )\Rightarrow \|u(x)\|_j<1$. \\
   En particulier : $(\sum\limits_{i\in I_0}^{} \|x\|_i)<\varepsilon \Rightarrow \|u(x)\|_j<\varepsilon $.\\
   Par homogénéité : $\|u(x)\|_j\le \varepsilon m^{-1} \sum\limits_{i\in I_0}^{} \|x\|_i$.\\
   \textbf{Réciproque :} On montre la continuité en 0 (donc en tout point par linéarité).\\
   On a $u(0)=0$. Soit $W$ un voisinage de 0 dans $F$. OPS $\exists J_0\subset J$ fini $\varepsilon >0, W=\{y\in F|\forall j\in J_0,\ \|y\|_j<\varepsilon  \} $ pour tout $j\in J_0$ on dispose de $C_j$ et $I_j\subset I$ finis tq $\|u(x)\|j\le \cdots$\\
   On pose $I_0=\bigcup_{j\in J_0}I_j$ et $\eta=\frac{\varepsilon}{\max C_j}\frac{\times_1}{\|I_0\|}>0$ et $V=\{x\in E|\forall i\in I_0,\ \|x\|_o<\eta \}$ est un voisinage de 0 et $\forall x\in V,\ \forall j\in J_0,\ \|u(x)\|_j\le C_j \sum\limits_{j\in J_0}^{} \|x\|_j<C_j\eta\|I_j\|\le \varepsilon $
\end{proof}

\begin{propriete}
    Soit $E$ un ev+lc muni d'une famille dénombrable de semi normes $(\|.\|_n)$. Alors la topo de $E$ est maitrisable pour la distance $d(x,y)=\sum\limits_{n\in \mathbb{N} }^{} \min(2^{-n}, \|x-y\|_n)$.
\end{propriete}
\begin{proof}
    Montrons que les bases de voisinage de l'origine $(B_j(0,\varepsilon )_{\varepsilon >0})$ et $\{x\in E| \forall i\in I_0,\ \|x\|_i<\eta \}, I_0$ fini $\varepsilon \eta>0$ sont équivalentes.\\
    Soit $\varepsilon >0$ et $N$ tq $2^{-N}<\varepsilon /3$. On considère $V=\{x\in E|\forall n<N,\ \|x\|_n<\frac{\varepsilon}{3N} \} $. Alors $\forall x\in V,\ d(x,0)<\sum\limits_{n=0}^{N-1} \frac{\varepsilon}{3N}+\sum\limits_{n=N}^{\infty} 2^{-n}=\varepsilon /3+2^{-N}*2\le \varepsilon $.\\
    Réciproquement : $V=\{x\in E|\forall n\in I_0,\ \|x\|_n<\eta \}$. Alors $V\subset B(0,\varepsilon )$ où $\varepsilon =\min(2^{-N-1},\eta)$ et finalement $\forall x\in B(x,0),\ \forall n\le N,\ \|x\|_n<\varepsilon \le \eta$.

\end{proof}


La topologie est engendrée par la base d'ouverts : $\{y\in E|\forall i\in I_0,\ |x-y|_i<\varepsilon  \} $ où $x\in E, I_0\subset I$ est fini et $\varepsilon >0$. Si on fixe $x$, on obtient une base de voisinage de $x$.

\begin{lemme}
    Un ev+lc $(E,\|.\|_i)$ est séparé ssi $\forall x\in E, \ (\forall i\in I,\ |x|_i=0)\Rightarrow x=0  $, \\
    ssi $\forall x\in E\backslash \{0\} ,\ \exists i\in I,\ |x|_i>0. $
\end{lemme}
\begin{proof}
    \begin{itemize}
        \item Si $\exists x\in E\backslash \{0\} ,\forall i\in I, \ |x|_i=0 $ alors $x$ appartient à une base de voisinage de $0$. $\{y\in E|\forall i\in I_0,\ |y|_i<\varepsilon  \} $ pour même conditions qu'avant donc l'espace n'est pas séparé.
        \item Si $\forall z\in E\backslash \{0\} ,\ |z|_i>0 $. Soit $x\neq y\in E$. Soit $i\in I$ tq $\underbrace{|x-y|_i}_{:=\varepsilon} >0$. Alors $\{z\in E\mid z-x|_i<\varepsilon /2\} $ et $\{z\in E\mid z-y|_i<\varepsilon /2\} $ sont des voisinages distincts de $x$ et $y$ donc l'espace est séparé.
    \end{itemize}
    On abrège evtlc séparé en evtlcs.
\end{proof}

Soit $(E,|.|_i))$ un evtlcs muni d'une famille dénombrable de semi normes.
\begin{itemize}
    \item On dit qu'elle est étagée si $\forall x\in E,\ (|x|_i) $ est croissante. On peut supposer, quitte à considérer $(|.|'_i)$ où $|x|'_i:=\max_{n\le i }|x|_n$ qui définit la même topo.
    \item On a la base d'ouverts $B_N(x,\varepsilon ):=\{y\in E|\forall n\le N,\ |y-x|_n<\varepsilon  \} =\{y\in E| |y-x|'_N<\varepsilon \} $ où $x\in E, N\in \mathbb{N} ,\varepsilon >0$.
    \item La topo est métrisable pour la distance $d(x,y)=\max_{n\in \mathbb{N} }\min(2^{-n}, |x-y|_n)$.
\end{itemize}

On note que $B_d(n, \eta)=\{y\in E| \forall n\in \mathbb{N} , \min(2^{-n},|x-y|_n)< \eta\} =\{y\in E| \forall n\le |\log_2\eta|,\ |x-y|_n<\varepsilon  \} $.\\
En effet $2^{-n}\ge \eta\Leftrightarrow -n\log_2\ge  \log_2\eta$.\\

On note que $B_d(x,\min(2^{-N}, \varepsilon ))\subset B_N(x,\varepsilon )$.\\
$B_{\left\lfloor |\log_2\eta| \right\rfloor}(x,\eta)\subset B_d(x, \eta)$

\begin{ex}
Fonctions non bornées :Soit $\Omega\subset \mathbb{R} ^d$ ouvert et $(\Omega_i)$ une suite d'ouverts tq $\bigcup_{n\in \mathbb{N} }\Omega_n =\Omega$ et $\forall n\in \mathbb{N} ,\ \overline{\Omega_n}\underbrace{\subset_C }_{\text{partie compacte de}} \Omega $.
\end{ex}

\begin{remarque}
    On peut poser $\Omega_n:=\{x\in B(0,n)|\forall y\in \mathbb{R} ^d\backslash \Omega,\ |x-y|>\frac{1}{n} \} $.
\end{remarque}

Pour tout $n\in \mathbb{N} , \alpha \in \mathbb{N} ^d$ et $f:\Omega\to R$ assez régulière, on pose $|f|_{n,\alpha }:=\sup_{x\in \overline{\Omega_n}}|\partial_\alpha f(x)$ où $\partial_{\alpha _1,\cdots, \alpha _d}f:=\frac{\partial ^{|\alpha|}f}{\partial_{\alpha_1}^{\alpha _1}\cdots\partial_{\alpha _d}^{\alpha _d} } $. Alors $\forall k\in \mathbb{N} , (C^k(\Omega),(|.|_{n,\alpha })^{|\alpha |\le k}_{n\in \mathbb{N} }  $. Est séparé et métrisable car $\mathbb{N} \times\mathbb{N} ^d$ est dénombrable.

\begin{ex}
    Classe $D(\Omega)$ des fonctions test : Soit $\Omega\subset \mathbb{R} ^d$ ouvert, $D(\Omega)=\{f\in \mathcal{C}^\infty (\Omega)|\sup f\subset _C\Omega\} $
\end{ex}

Pour tout $w\,eta\in C^0(\Omega,\mathbb{R} _+)$ on pose sur $f\in D(\Omega)$. $|f|_{w,\eta}:=\sup_{x\in \Omega, \alpha \le \eta(x)}|w(x)| |\partial^\alpha f(x)|$. \\

Alors $D(\Omega)$ est un ouvert et evtlc :) .

L'espace $D^*(\Omega)$ des formes linéaires continues sur $D(\Omega)$ est appelé espace des distributions. \\
$\forall \varphi \in D^*(\Omega),\ \exists w,\eta\in C^0(\Omega,\mathbb{R} ^+), \ \forall f\in D(\Omega),\ |\underbrace{\varphi (f)}_{\text{parfois noté}\\<\varphi ,f>_{D^*\times D}}|\le \underbrace{|f|_{w, \eta}}_{\substack{\text{En principe, } C\max_{1\le i\le I}|f|_{w_i , \eta_i} \\\text{ mais on peut se ramener à une seule}}}$

Une distribution $\varphi $ est d'ordre fini $k\in \mathbb{N} $ si $\exists w\in C^0(\Omega,\mathbb{R} _+),\ \forall f\in D(\Omega),\ |\varphi (f)|\le |f|_{w,k} $.
\begin{ex}
    Distribution d'ordre fini :
    \begin{itemize}
        \item Masse de Dirac $\varphi (f)=f(0)$ est d'ordre 0
        \item Si $g\in L_{loc}(\Omega),$ alors $\varphi (f):=\int_\Omega fg$ est une distribution. \\
            Si $d=1$, $\varphi $ est d'ordre 1. En effet soit $G$ une primitive de $g$ s'annulant en 0 (si 0$\in \Omega)$. \\Alors $\int_{t_0}^{t_1}f(t)g(t)dt=[fG]_{t_0}^{t_1}-\int_{t_0}^{t_1}f'(t)G(t)dt$. On choisit $t_0,t_1$ tq $supp(f)\subset [t_0,t_1].$ \\Alors $|\varphi (f)|=\int_{t_0}^{t_1}|f'(t)| |G(t)|dt$ On pose $\eta=1 $, $w(t)=z(t)\sup|G(s)|$ (à vérifier)
        \item $\varphi (f)=f'(0)$ distrib d'ordre 1
        \item $\varphi (f)=\sum\limits_{n\in \mathbb{N} }^{} f^{(n)}(n)$ d'ordre $\infty $ avec $\eta=Id, w=Id$.
        \item  Classe de Schwartz (compatible avec la transformée de Fourier et métrisable) : on pose pour tout $n\in \mathbb{N} , \alpha \in \mathbb{N} ^d,f\in C^\infty (\mathbb{R} ^d)$, $|f|_{n, \alpha }:=\sup_{x\in \mathbb{R} ^d}(1+|x|^2)^{\frac{n}{2}}|\partial_\alpha f(x)|$. Toutes les dérivées décroissent plus vite que n'importe quelle paissance négative. evtlc métrisable séparable\ldots
        \item Topo faible est * faible : soit $E$ un evtlc * la topo faible sur $E$ est définie par les semi normes $x\in E\mapsto |l(x)|$ où $l\in E^*$. C'est la topo la plus faible qui rend les formes linéaire continue. La séparation nécessite de construire des formes linéaires et découle du théorème de Hahn-Banach. Pas métrisable (exo) sauf en dim finie.
        \item topo * faible sur $E^*$ est def par la famille de semi normes $l\in E^*\mapsto |l(x)|$ est séparé (en effet pour $l\in E^*$ sur lequel toutes ces semi normes s'annulent alors $l$ est la fonction nulle ie $l=0$.) et pas métrisable sauf si dim finie.
    \end{itemize}
\end{ex}
\begin{proposition}
    Métrisabilité de la boule unité pour la topo * faible :\\
    Soit $E$ un evn séparable, soit$(x_{n})$ une suite dense dans $B'_E(0,1)$ et soit $B:=B'_{E^*}(0,1)$. Alors la topologie * faible sur $B$ est métrisable poir la distance $d(u,v):=\max_n\min(2^{-n}, |u(x_{n}-v(x_{n}|)$
\end{proposition}

\begin{remarque}
    On pourrait remplacer $B$ par n'importe quelle partie bornée de $E^*$.
\end{remarque}

\begin{proof}
    Soit $u\in B$ et un voisinage de $u$ pour la distance $d_{|B\times B}$ de la forme $B_d(u, \eta)=\{v\in B| \forall n\le |\log_2\eta|,\ |u(x_{n})-v(x_{n})|<\varepsilon\} $.\\
    \textbf{Réciproquement :} soit $u\in B$ et soit un voisinage de $u$ pour * faible de la forme $\{v\in B|\forall 0\le k\le K,\ |u(y_{k})-v(y_{k}|<\varepsilon  \} $. OPS $\|y_k\|\le 1$ quitte à cibsidérer $y_k/\alpha $ et $\varepsilon \alpha $. Soit $n_0, \cdots, n_K$ tels que $\|x_{n_k}-y_k \|\le \varepsilon /2$ avec $\alpha =\max(1, \max_{0\le k\le K}\|y_k\|)$. Soit $N:=\max(n_0,\cdots,n_K$ et $\eta=\min(2^{-N},\varepsilon /2)$. Alors $B_d(u, \eta)\cap B\subset \{v\in B| \forall n\le N,\ |v(x_{n})-u(x_{n})|<\varepsilon /3 \} =V$. Soit $v\in V$ et $k\le K$ alors $|v(y_k)-u(y_k)|\le |v(y_k)-v(x_{n_k})|+|v(x_{n_k})-u(x_{n_k})|+|u(x_{n_k})-u(x_k)|\le \|v\|_{E^*}\|y_k-x_{n_k}\|+|v(x_{n_k})-u(x_{n_k})|+\|u\|_{E^*}\|y_k-x_{n_k}\|\le 1*\varepsilon /3+\varepsilon /3 +1*\varepsilon /3<\varepsilon $ donc $V\subset V_0$ on a bien une base de voisinage fournie par la métrique.
\end{proof}
